Das Dezernat für Internationale Beziehungen (Abteilung Akademischer Austausch) der Universität Tübingen betreibt ein umfangreiches Austauschprogramm mit vielen Unis weltweit. Zusätzlich gibt es noch einige direkte Kontakte der Informatik zu anderen europäischen Unis im Rahmen des ERASMUS-Programms.

Wie wichtig ein Auslandsaufenthalt für die persönliche Erfahrung, das fachliche Weiterkommen, die Selbständigkeit, die eigene Motivation und die Erweiterung des gesellschaftlichen Horizonts etc. sein kann, sollte eigentlich nicht weiter betont werden müssen. Jedem sei empfohlen: Geh'\ ins Ausland!

Da die Deadline für eine Bewerbung meist ein Jahr vor dem gewünschten Abreisetermin liegt, solltet ihr euch rechtzeitig darüber klar werden ob und wohin ihr wollt.  Lieblingsländer der Studenten sind eindeutig USA/Kanada einerseits und Australien/Neuseeland andererseits -- über den Akademischen Austausch sind in den letzten Jahren meist 80 \% der Bewerber in ihr Wunschland gekommen. Die Chancen stehen also eigentlich relativ gut, insbesondere da Naturwissenschaftler immer gerne genommen werden.

Im Rahmen des ERASMUS-Programms bietet die Fakultät auch auf unsere Initiative hin inzwischen Aufenthalte in viele europäische Länder an, darunter Frankreich, Italien, Schweden und Spanien. Der Austausch ist dabei über direkte Kontakte zwischen den Lehrstühlen der Unis organisiert. Das hat aber nicht unbedingt Bedeutung für eine dortige Fächerwahl oder gar den Studiengang, es regelt einzig die Anzahl der Plätze. Allerdings sehen die Lehrstühle es manchmal nicht gern, wenn ihr komplett eurem Nebenfach, welches nicht im Informatik - Lehrstuhl liegt, nachgeht. 
Bei Interesse solltet ihr euch daher direkt an die teilnehmenden Profs wenden -- eine mehr oder weniger aktuelle Liste findet ihr auf der Website des Akademischen Austauschs (s.u.). Falls euer Wunschland nicht über einen Prof aus unserem Institut erreichbar sein sollte, könnt ihr auch versuchen einen der Restplätze im Programm einer anderen Fakultät zu bekommen. Die Chancen dafür stehen oft gar nicht schlecht, weil nur wenige Studenten überhaupt nachfragen. Als Vorlauf reicht teilweise auch nur ein halbes Jahr, dafür ist die Organisation mitunter sehr chaotisch (Studenten haben teilweise auch bis zur Abreise noch keine schriftliche Zusage). Da hilft Beharrlichkeit und Fax.  Im Erasmus - System wird teilweise auch für eure Unterkunft gesorgt. 

Wer auf Bachelor studiert, sollte sich bezüglich des Zeitpunkts seines Austauschs und der anrechenbaren Leistungen unbedingt vorher bei der Fakultät bzw. dem zuständigen Prof beraten lassen! Die späteren Semester im Bachelor und das zweite Studienjahr im Master sollten hier besonders geeignet sein.\\
Grundsätzlich kann man sich für die Zeit im Ausland beurlauben lassen. Es ruhen dann alle Prüfungsfristen (die betreffenden Semester hat es quasi nie gegeben). Trotzdem können natürlich alle Leistungen im Ausland entsprechend den Regelungen in der Studienordnung angerechnet werden.

Ein Wort muss auch noch zum TOEFL-Test, dem Test of English as a Foreign Language, verloren werden. Insbesondere für einen Austausch mit Nordamerika wird dieser Test fast zwingend gefordert. Anmelden kann man sich ausschließlich über den Organisator ETS auf deren Website \url{https://www.ets.org/toefl}. In Tübingen gibt es derzeit zwei Testcenter, eines wird vom Fachsprachenzentrum betrieben, das andere wurde von Mitgliedern der FSI ins Leben gerufen. Weitere Informationen bekommt ihr auf unserer Website unter \url{https://www.fsi.uni-tuebingen.de/studium/toefl}.
Im Erasmus- Austausch werden keine Sprachtests erwartet.

Finanziell werden euch beim Austausch über den Akademischen Austausch erstmal die Studiengebühren an der Partneruni erlassen, was bei einigen Universitäten sehr viel ausmacht.  Wer BAFöG bekommen kann (für's Ausland geht das noch etwas einfacher als für's Inland), bekommt unter Umständen noch deutlich mehr Kosten erstattet. Hier solltet ihr euch aber noch einmal speziell beraten lassen. Wer über Erasmus geht, bekommt die Differenz zwischen den Lebenshaltungskosten in Tübingen und denen im Austauschort zum größten Teil über ein Stipendium bezahlt. 

Immer gibt es natürlich die Möglichkeit sich seinen Austausch selbst zu organisieren (enorm aufwendig und sehr teuer!) oder sich für ein Stipendium von dritter Seite, z. B. dem DAAD (Deutscher Akademischer Austauschdienst) oder das Baden-Württemberg Stipendium, zu bewerben.  Das ist aber natürlich wesentlich aufwändiger als an einem Programm der Uni teilzunehmen und erfordert viel Eigeninitiative. Der Akademische Austausch kennt alle wichtigen Stipendiengeber - einfach mal in der regelmäßigen Sprechstunde (derzeit unter der Woche außer mittwochs zwischen 9 und 12 Uhr) nachfragen.

Für alle Interessenten wichtig sind die Info-Veranstaltungen des Akademischen Austausches, die auf deren Seite angekündigt werden.
\medskip

Die Abteilung Akademischer Austausch befindet sich in der \\
Nauklerstr. 2, Tel.: 29-76448 und ist im Internet unter\\
\url{https://www.uni-tuebingen.de/international}\\
zu erreichen.
