Der Bachelorstudiengang Medieninformatik dient dem Ziel, den Studierenden die für einen ersten berufsqualifizierenden Abschluss notwendigen Kenntnisse in den Anwendungsbereichen Internet, Mensch-Computer-Interaktion und digitale Medien zu vermitteln. Das  Spezifische  der  Medieninformatik besteht darin, dass neben der Technik der Mensch, der mit dem oder mittels des Computers kommuniziert, in besonderer Weise ins Blickfeld genommen wird.

Kern des Studiums ist eine solide Informatikausbildung. Im ersten Semester hört ihr neben der mathematischen und informatischen Einführungsveranstaltungen auch \textbf{Einführung in die Medienwissenschaft}. Die  nächsten  zwei  Semester  werden  von  Pflichtmodulen  in  Informatik,  Mathematik  und  Medieninformatik  bestimmt. Ab dem 4. Semester habt ihr, neben \textbf{Graphischer Datenverarbeitung} und \textbf{Algorithmen} nur noch Wahlpflicht Veranstaltungen, in denen hier jede Vorlesung der Informatik/Medieninformatik belegen könnt die ihr noch nicht gehört habt.

\begin{itemize}

\item Die Vorlesung \textbf{User-Interface-Design} führt die beiden alten Vorlesungen \textbf{Human Computer Interaction} und \textbf{Usability Engineering}, welche große Überschneidungen aufwiesen, zusammen. Sie beschäftigt sich mit Benutzerschnittstellen und möglichen Problemen in der Bedienung für behinderte und ältere Menschen (Wie mache ich mein Programm für ältere oder behinderte Menschen zugänglich?). Außerdem lernt ihr mit welchen Methoden ein Projekt angegangen werden kann. Die in der Vorlesung verwendeten Folien stehen online zur Verfügung (auch mit Platz für eigene Notizen).

\item Eine weitere Pflichtveranstaltung im ersten Semester ist \textbf{Einführung in die Medienwissenschaft}, hier lernt ihr unter anderem die Theorien und Methoden der Medienwissenschaft, Kommunikationsmodelle und Ethik der Medien kennen. Dies gibt euch eine solide Basis in der Medienwissenschaft 

\item In der Vorlesung \textbf{Grundlagen der Multimediatechnik} werden die Grundlagen für die Verarbeitung digitaler Audio- und Videodaten behandelt. Es werden Kompressionsverfahren vorgestellt und moderne Speichermedien für die Aufzeichnung und Wiedergabe von Multimediadaten diskutiert.

\item Im zweiten Semester wird die Vorlesung \textbf{Grundlagen Internettechnologien} angeboten. Im Wesentlichen werden hier die aus Grundlagen der Mensch-Computer-Interaktion bekannten HTML und CSS-Kenntnisse erweitert und ein Überblick über wichtige Skriptsprachen im Web wie PHP, Perl und JavaScript gegeben. Zusätzlich wird mit MySQL ein Einblick in Datenbankapplikationen gegeben. Die Übungen sind nicht sonderlich zeitaufwändig (bittet am besten darum in den Übungsgruppen Medieninformatiker und Medienwissenschaftler zu trennen, da die Informatiker meist deutlich schneller durch sind) und bestehen hauptsächlich aus dem Erstellen von einfachen Websites. Am Ende wird das Gelernte in einem kleinen Projekt, welche in die Note mit einfließt, praktisch angewendet.

\end{itemize}

Als Orientierungsprüfung müsst ihr die Veranstaltungen "`Mathematik I"' oder "`Mathematik II"' sowie "`Informatik I"' oder "`Informatik II"' bestehen.

%Das Prüfungssekretariat Medieninformatik ist bei Frau Hallmayer im Raum B118 im Sand 13.  Die Öffnungszeiten
%sind Mo-Fr 10:00-12:00 Uhr und Di, Do 14:00-16:00.

%Für die Betreuung der Informatik in den Medienwissenschaften sind Julian Wangler, M.A.\footnote{\email{julian.wangler@uni-tuebingen.de }} und Dipl. Medienwiss. Felix Reer\footnote{\email{felix.reer@uni-tuebingen.de }} zuständig.

Unter der Adresse \url{http://www.medieninformatik.uni-tuebingen.de/} stehen ein paar weitere Informationen zum Studiengang zur Verfügung.
