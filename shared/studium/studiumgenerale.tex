% studium/studiumgenerale.tex

Das Studium Generale ist für Studenten und für Leute außerhalb der Universität
  gedacht und deckt -- mit mindestens einer Veranstaltung pro Tag! -- alle
  möglichen Gebiete ab.  Die Vorlesungen finden abends statt, meistens ab 18
  c.t., wo "`normale"' Leute Zeit haben.  Jedes Thema hat dann seinen festen Zeitplatz im
  Ablauf der Woche.

Die Themen sind  sehr interessant, und es ist üblicherweise
  zu jedem Termin ein anderer Dozent (meist Profs von verschiedensten Unis)
  dran, was jedes Mal eine hervorragende, gut vorbereitete Vorlesung
  garantiert.  Nicht zuletzt deshalb werden die Studium-Generale-Vorlesungen
  dann doch hauptsächlich wieder von Studis und Profs besucht, die in
  andere Fächer hinein schnuppern wollen.

Das Programm liegt jeweils am Anfang jedes Semesters in Form blauer
  Heftchen überall aus  und wird auch an einigen Pinnwänden  bekanntgegeben.
