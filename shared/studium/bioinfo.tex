
Die Bioinformatik in Tübingen ist inzwischen schon einige Jahre etabliert
und durch das Zentrum für Bioinformatik (ZBIT) sowie die Professuren für Simulation
biologischer Systeme und Algorithmen der Bioinformatik an der Fakultät gut verankert.

Da alle Prüfungsleistungen studienbegleitend erworben werden, ist insbesondere in den ersten Semestern dann ein gutes Zeitmanagement gefragt. Es gibt auf jeden Fall weniger stressige Studiengänge!

Hier jetzt zunächst einige Hinweise zu den Veranstaltungen der ersten zwei Jahre:

\begin{itemize}

\item Die Vorlesung \textbf{Biomoleküle und Zelle}, welche erst zum Wintersemester 2007 eingeführt wurde, wird vom Vorlesungsverzeichnis wie folgt beschrieben:\\
"`Die Vorlesung gibt einen kurzen Abriss der biochemischen Grundlage des Lebens, führt in die grundlegenden Strukturen eukaryotischer und prokaryotischer Zellen ein und
beschreibt die Prinzipien von Zellwachstum und -vermehrung. Sie erläutert die molekulare Basis der Erbinformation, den Fluss der genetischen Information von DNA zu Protein
und die Konsequenz von Mutation und Rekombination. Neben einem Einblick in die Grundlagen der Bakterien und Viren-Genetik wird eine Einführung in die Gentechnik gegeben."'\\
Die Vorlesung wird von Professor Nordheim gehalten (oder zumindest organisiert), der in den Zellbiologievorlesungen als sehr umgänglicher und freundlicher Mensch aufgefallen ist.
Sollte es irgendwelche Probleme mit der Vorlesung geben, werdet Ihr euch sicher gerne an ihn wenden dürfen.

Zur Vorlesung findet außerdem ein Praktikum statt. Ihr solltet daran unbedingt teilnehmen, auch wenn es sich mit anderen Veranstaltungen überschneidet.

     Die Vorlesungen Zellbiologie, Mikrobiologie und Genetik sind zusammen mit den Biologen und
     parallel zu dem Praktikum. Von allen Gebieten gibt es auch ein Skript, das im Internet zum Download angeboten wird.
     Besonders aufzupassen gilt es bei der Vorlesung von Prof. Götz (Mikrobiologie), da
     dieser mit Vorliebe sehr detaillierte Fragen in der Klausur stellt. Im letzten Jahr hat er jedoch
     am Ende jeder Vorlesungen ca. 10 Fragen aufgelegt und mit den Studenten besprochen.
     Fast alle seiner Fragen in der Klausur wurden hier bereits diskutiert!
     Insgesamt sollte man die Klausuren nicht unterschätzen, da sie während
     dem Semester stattfinden und so die Zeit zum Lernen etwas knapp ist. Alte Klausuren gibt es bei der
     Fachschaft Biologie zum kopieren (einfach im E-Bau 1. Stock vorbeischauen) oder auch in unserem Protokollsystem zum Download.

\item In der \textbf{Tierphysiologie} (Neurobiologie) gibt es eine Vorlesung und ein
     Praktikum im bzw. nach dem drittem Semester. Für den Fall, dass das Praktikum wie immer unter der Leitung von Prof. Ilg abläuft, folgen hier die Tipps aus den letzten Jahren:\\
     Für das Praktikum solltet
     ihr euch die kompletten ersten Semesterferien im Frühjahr freihalten, die zusammen
     mit den Informatikklausuren nahezu komplett ausgefüllt sein werden!
     Für das eigentliche Praktikum gibt es ein zusätzliches Skript, mit
     dem man sich sorgfältig vorbereiten sollte (auch andere Bücher
     dazu lesen, v.a. "`Neurowissenschaften"' von Bear), da es vor jedem Versuch eine kleine
     Prüfung in Form eines Gruppengesprächs gibt.  Bei schlechter
     Vorbereitung werden ganze Gruppen nach Hause geschickt (Die
     Begründung hierfür ist, dass für einen schlecht vorbereiteten
     Studenten kein Tier getötet werden soll\textellipsis).  Auch hier gilt
     wie für alle Praktika, dass ihr euch einen großen Gefallen
     tut, wenn ihr alles gut dokumentiert ("`Wozu gehört diese
     komische Kurve nochmal?"') und die Protokolle so schnell wie
     möglich schreibt -- man vergisst
     sonst alles.  Die erste Version müsst ihr allerdings sowieso
     schon beim nächsten Praktikumstermin zwei Tage später
     abgeben. Es \textbf{empfiehlt sich} vor den Prüfungen einige alte Klausuren
     anzuschauen (ein Großteil der Fragen war schon früher mal dran!).
     Dann ist die Tierphysiologie allerdings gut machbar.

 \item Die \textbf{Chemie-Vorlesung} im ersten Semester besteht aus zwei Teilen,
     Anorganische und Allgemeine Chemie, sowie der Organische Chemie. Den ersten
     Teil wird Herr Schweda, den zweiten Herr Speiser lesen.
     Leider gibt es jedoch keine Übungen oder ähnliches und man tut deshalb nicht so
     viel für diese Vorlesung, wie es eigentlich nötig wäre.
     Dann steht man vor der Prüfung da
     und kann gerade nochmal von vorne anfangen. Also: Gleich gut
     mitarbeiten erleichtert einiges. Auch wenn man am Anfang denkt
     ist ja ganz einfach, es kommen Themen bei denen man schon was tun
     muss.
     Von Herrn Schweda gibt es ein ausführliches Skript im Internet, das auf seiner Seite
     direkt zum Download angeboten wird. Herr Speiser hat auf seiner Homepage einen Link,
     der euch zu einem kleinen Teil seines Skripts führt, es existieren aber auch (inoffizielle)
     Mitschriebe Eures Vorgängerjahrgangs.
     Grundsätzlich sind die Chemievorlesungen jedoch sehr kompakt, auch im
     Vergleich zu "`richtigen"' Chemievorlesungen. Praktisch wird etwas mehr
     Wissen als in einem Chemie-LK (oder wie auch immer das in euren Bundesland heißt)
     behandelt wurde vermittelt. Das heißt für Chemieabwähler: Ihr werdet
     erfahrungsgemäß von der Vorlesung erschlagen -- zum Nachbereiten bleibt
     wegen der zeitaufwändigen Übungen in den Informatikvorlesungen kaum Zeit.
     Ihr solltet euch daher dann etwa einen Monat vor der Prüfung freihalten...

\item Diese beiden Vorlesungen werden zusammen mit einem Kompaktpraktikum nach dem
     zweiten Semester im Rahmen des Moduls "`Chemie I"' geprüft. Wie in den
     letzten Jahren wird das Praktikum für euch vermutlich von Prof. Zeller
     organisiert werden und in der ersten Woche nach Ende der Vorlesungszeit
     des zweiten Semesters stattfinden. Anschließend ist ein recht umfangreiches
     Protokoll auszuarbeiten, welches übrigens handschriftlich erstellt werden muss.

\item Außerdem gibt es im zweiten Semester eine \textbf{Biochemie-Vorlesung}
     mit den Biologen zusammen. Sie wird von Professor Nürnberger gehalten. Seine Folien sind hilfreich und er hält eine
     ansprechende Vorlesung. Anschließend wird es eine schriftliche
     Prüfung geben, deren Note dann zusammen mit jener aus der Physikalischen Chemie (s.u.)
     die Gesamtnote des Moduls Chemie II ausmachen wird. Auch hier empfiehlt es sich mal in Altklausuren reinzuschauen.


\item Im fünften Semester findet außerdem noch eine Vorlesung
     \textbf{Physikalische Chemie} bei Prof. Gauglitz statt, zusammen mit Pharmazeuten
     und Geoökologen. Er präsentiert viele lustige Applets, wenn sein Laptop
     funktioniert und lässt einen auch mal früher gehen -- der
     Formelschlacht ist allerdings kaum beizukommen, das Skript, das es im
     Internet gibt (\url{http://barolo.ipc.uni-tuebingen.de/lehre/portal}) ist ein wenig
     chaotisch, so lohnt der Vorlesungsbesuch zur Strukturierung in jedem Falle.
     Herr Gauglitz lässt eine schriftliche Prüfung durchführen. Damit man nicht ganz unvorbereitet
     in die Klausur geht, wird ein Tutorium angeboten.
     Den Erfahrungen nach ist dieses sehr anspruchsvoll, aber ohne dieses Tutorium ist die Klausur
     kaum machbar. Auch zu dieser Vorlesung findet ein Praktikum statt.
     Für dieses Praktikum gibt es ein Eingangstestat im Internet und
     Vorprotokolle (Erarbeitung der wichtigsten Inhalte und Beantwortung einiger Fragen).  Ein
     Vorprotokoll macht i.a. mehr Arbeit als ein normales Protokoll.  Hat man es aber selbst geschrieben
     (oder zumindest die Inhalte verstanden), dann ist das normale Protokoll eine Sache von einer Stunde
     (am besten mal Protokolle von Chemikern ansehen, wie die das machen). Ihr solltet unbedingt die
     verschiedenen Abgabefristen für Testat, Vorprotokoll, etc. beachten!
     Am Schwarzen Brett des Lehrstuhls (B-Bau, 8. Stock) befindet sich zudem eine genaue Anweisung, wie viel
     Zeit man für die einzelnen Korrekturversuche je Protokoll hat.

\item Im zweiten Semester findet außerdem seit 2010 eine Ringvorlesung \glqq Einführung in die Bioinformatik\grqq \ statt. Sie gibt euch einen ersten Einblick in euer Studienfach,
indem viele Themenbereiche aus der Bioinformatik in einer 2-stündigen Vorlesung von verschiedenen Professoren vorgestellt werden.

\end{itemize}

Als Orientierungsprüfung müsst ihr die Veranstaltungen "`Mathematik I"' oder "`Mathematik II"' sowie "`Informatik I"' oder "`Informatik II"' bestehen.

Im letzten Jahr des Bachelorstudiengangs Bioinformatik könnt ihr dann eine eigene Ausrichtung wählen.
Hier werden Molekularbiologie/Genetik, Neurobiologie, Pharmazie und Biochemie/Chemie angeboten. An
unserer Fakultät sind sicherlich die Themen 1 und 4 am stärksten vertreten, alle anderen
Studienrichtungen sind aber mit etwas mehr Organisationsaufwand ebenfalls gut studierbar. Neurobiologen
können z.B. teilweise die Veranstaltungen der Graduate School of Neural Science besuchen.

Das Master-Programm besteht im Kern aus Veranstaltungen rund um die Vorlesung \textbf{Algorithmen der
Bioinformatik}. Dazu gibt es auch ein Praktikum und ein Seminar. Daneben ergänzt ihr das Wissen aus
dieser Vorlesung um eigene Schwerpunkte in der Biologie (wie oben) und Informatik.

%Das Prüfungssekretariat Bioinformatik ist bei Frau Weber im Raum B316 im Sand 13.  Die %Öffnungszeiten
%sind Di, Mi, Do 9:30-11:30.


