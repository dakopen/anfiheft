Neben den rein fachlichen Vorlesungen m\"ussen inzwischen auch Leistungen in sogenannten \"Uberfachlichen berufsfeldorientierten Kompetenzen (fr\"uher Schl\"usselqualifikationen) erbracht 
werden. Je nach Studiengang ist der Anteil unterschiedlich gro\ss 
 (beispielsweise sind in der reinen Informatik lediglich 6 LP, ansonsten 12 LP zu erbringen). Die urspr\"ungliche Idee hinter den 
\"UbK ist die Auseinandersetzung mit gesellschaftlichen 
Themen sowie der Erwerb au\ss{}erfachlicher Kompetenzen. Deswegen finden sich 
auch Verstanstaltungen wie  "`Einf\"uhrung in das Recht f\"ur Informatiker"' 
oder "`Grundlagen wissenschaftlichen Arbeitens"' in diesem Bereich.

Falls man sich andere Veranstaltungen anrechnen lassen m\"ochte, muss man 
theoretisch einen Antrag an den Pr\"ufungsausschuss stellen; praktisch ist 
das meistens nicht n\"otig. Um den Vorsitzenden des 
Pr\"ufungsausschussvorsitzenden der Informatik zu paraphrasieren: 
Informatikstudenten erwerben im Studium alle wichtigen Kompetenzen und brauchen 
keine berufsvorbereitenden Veranstaltungen; dementsprechend ist effektiv jede 
benotete Veranstaltung anrechenbar (ausgenommen Sport). Insbesondere f\"ur 
Studenten der Informatik, Medieninformatik und Bioinformatik ist zu beachten, dass im Bereich der 
\"Uberfachlichen berufsfeldorientierten Kompetenzen mindestens ein Pro-Seminar zu belegen ist. Im Master 
Informatik sowie Medieninformatik ist ein Seminar zu belegen.

% Informatik zu studieren ist eine nette  Beschäftigung.   Eines Tages 
%   allerdings wird der frisch ausgebildete
%   Informatiker in das Berufsleben einsteigen.  Spätestens dann beginnt das
%   eigene Handeln, Bedeutung für andere Menschen zu haben -- und sei dies zu
%   Beginn nur der eigene Chef.
% 
% In der Geschichte der Informatik hat die Förderung der Forschung durch das 
% Militär eine große Rolle gespielt - ohne diese Förderung wäre die Informatik in 
% ihrer heutigen Form nicht denkbar.
%   Viele der interessantesten und anspruchsvollsten Aufgaben
%   finden sich auch heute noch im militärischen Bereich.
% 
% Im zivilen Bereich mag einem George Orwells  "`1984"' vor
%   Augen stehen.  In England werden heute schon ganze Städte von
%   Kameras überwacht und je mehr wir uns in Richtung
%   "`Informationsgesellschaft"' entwickeln, desto mehr stellt sich die
%   Frage nach Daten -- und Persönlichkeitsschutz. Insbesondere vor dem 
% Hintergrund des "`War against terrorism"' hat die staatliche Überwachung der 
% Bevölkerung in vielen Staaten erheblich zugenommen.
% 
% Szenarien wie im Kult-Roman "`Neuromancer"' von William Gibson mögen auf den
%   ersten Blick sehr weit hergeholt sein, aber so ist man z.B. heute schon
%   dabei, eine künstliche Netzhaut zu bauen.  Das bedeutet, man nimmt --
%   computerunterstützt -- direkten Einfluss auf das Nervensystem.  Wie weit ist
%   dann noch der Schritt zur direkten Einflussnahme auf das Gehirn?  
% 
% Welche Art der Forschung wollen wir? Müssen wir als Informatiker für die Folgen 
% der von uns entwickelten Technologien Verantwortung übernehmen?
% 
% Um  angehenden Informatikern für solche Probleme zu sensibilisieren, gibt es
%   das Modul "`Schlüsselqualifikationen"'.
%   In den Schlüsselqualifikationen sollte man sich also mit den 
% Berührungspunkten zwischen der Informatik und Gesellschaft befassen. Dazu 
% gehören einerseits die Folgen der
%   Informatik(-forschung) für unser tägliches Leben.  Andererseits
%   wird aber auch die Einflussnahme der Gesellschaft auf die Informatik
%   untersucht.
% 
%  Seit einiger Zeit bieten aber praktisch alle Fakultäten auch eigene 
% Veranstaltungen an. Zudem können nun auch Veranstaltungen vom 
% Fachsprachenzentrum und Career Service besucht werden (auch wenn der 
% eigentliche Zweck des Moduls durch z.B. einen Spanischkurs ad absurdum geführt 
% wird). Wer sich andere als im Modulhandbuch angegebene Veranstaltungen als 
% Schlüsselqualifikation anrechnen lassen möchte, sollte sich die jeweilige 
% Veranstaltung vorab vom Prüfunssausschuss
% absegnen lassen.
% 
% Für die Informatik und Medieninformatik Studenten fallen auch das Seminar bzw. 
% Proseminar in das Modul Schlüsselqualifikationen. Dafür sind dann aber 
% insgesamt mehr LP für dieses Modul zu erwerben als bei den Bioinformatikern.
% 
% 
% Schon vor vielen Jahren formierten sich Informatik-spezifische
%   Berufsverbände.  Die \emph{Association for the Computing Machinery} (kurz:
%   ACM) in den USA und die \emph{Gesellschaft für Informatik} (kurz: GI) hier
%   bei uns in Deutschland.  Das Hauptziel dieser Verbände ist es, die
%   Informatik als Ganzes voranzubringen.  Doch haben beide
%   Verbände inzwischen erkannt, dass es eine informatikspezifische Ethik braucht.
%   Schaut doch mal rein:
%   \url{http://www.gi-ev.de/}.
% Neben diesen "`großen"' Berufsverbänden gibt es noch "`kleinere"', für
%   die das oben beschriebene Dilemma viel mehr im Mittelpunkt ihres Interesses
%   steht.  Dies ist zum einen das \emph{Forum InformatikerInnen für Frieden und
%   gesellschaftliche Verantwortung e.V.}
%   (kurz: FIfF, \url{http://www.fiff.de/}) hier in Deutschland und
%   die \emph{Computer Professionals for Social Responsibility} (kurz: CPSR) in
%   den USA.
