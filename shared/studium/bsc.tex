
Zum Wintersemester 2015/16 wurden die Pr\"ufungsordnungen aller Informatik-Studieng\"ange geändert.
Hiermit haben sich die Punktezahl der einzelnen Veranstaltungen, aber auch gravierendere Dinge wie die
Prüfungsmodalitäten und der Übungsbetrieb der einzelnen Veranstaltungen geändert. 
Es kann also vorkommen, dass manche Infos auf den folgenden Seiten noch fehlerbehaftet sind.
Falls ihr Fehler entdeckt, schreibt uns doch einfach an \email{fsi@fsi.uni-tuebingen.de}.

Hier noch einmal kurz und knapp die wichtigsten Dinge:

\begin{itemize}
\item Bis ca. 2 Wochen vor den Prüfungen müsst ihr euch verbindlich für alle Veranstaltungen anmelden,
  deren Klausur ihr mitschreiben wollt. %Dazu habt ihr ab Vorlesungsbeginn vier Wochen Zeit.
  Meistens ist dies über ein Onlinesystem -- das CAMPUS -- möglich, für manche Veranstaltungen müsst ihr noch persönlich
  ein Formular in euer Prüfungssekretariat tragen.

%\item Es gibt die unangenehme Möglichkeit Maluspunkte zu sammeln. Hat man mehr als sechs davon
%  zusammen, kann man sich ein neues Betätigungsfeld für die nächsten Jahre suchen. Wie kommt man in den Genuss dieser
%  Punkte? Ganz einfach, %zwei Wege stehen zur Auswahl: 1. Jede Prüfung darf man zwei Mal angehen, für jeden
  %weiteren Versuch wird euch jeweils ein Punkt gutgeschrieben. 2. 
%  Besteht ihr pro Semester weniger als 15 LP, bekommt ihr einen dieser Punkte geschenkt.
%  Folglich müsst ihr im Schnitt wenigstens Vorlesungen im Umfang von 15 LP pro Semester belegen.
%  Wie viele LP es je Vorlesung gibt, steht im Modulhandbuch.

\item Bis zum Ende des zweiten Semesters müsst ihr die Orientierungsprüfung ablegen. Sie umfasst je nach Studiengang zwei Vorlesungen.
  Klappt das bis zum Ende eures ersten Jahres nicht und ihr
  könnt keinen besonderen Grund dafür angeben, werdet ihr euch wohl oder übel nach
  etwas anderem für die nächsten Jahre umsehen müssen. Für beide Prüfungen gilt übrigens eine Besonderheit:
  Sie dürfen nur \textbf{einmal} wiederholt werden.
 
%\item Bis zum Ende des siebten Semesters müsst ihr die Zwischenprüfung abgelegt haben. Sie ist in der 
%	Prüfungsordnung genauer beschrieben und entspricht in etwa dem früheren Vordiplom. Zum Angeben bei
%  der Oma wird euch dann auch auf Wunsch ein entsprechendes Zeugnis darüber ausgestellt.

\end{itemize}

% WS2015/16 prüfungsanmeldung korrigiert, zwischenprüfung und maluspunkte rausgestrichen, disclaimer eingesetzt - tim