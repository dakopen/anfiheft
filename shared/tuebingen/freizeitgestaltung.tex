%\corona

\subsubsection*{Schwimmbäder}
In Tübingen gibt es drei Schwimmbäder, zwei Hallenbäder und ein Freibad. Die Hallenbäder wechseln sich mit ihren Öffnungszeiten ab. Wann welches Schwimmbad geöffnet ist, findet ihr auf der Website des SWT\footnote{\url{https://www.swtue.de/baeder/}}. Im Hallenbard Nord auf dem WHO ist außerdem eine Sauna untergebracht. Diese ist jedoch nicht sehr groß, und relativ teuer. Als Alternative bietet sich die Therme in Bad Urach (ca. 40min mit der Bahn) an.	%TODO insert \link{}{}?

\subsubsection*{Landestheater}
Das Landestheater(LTT)\footnote{\url{http://www.landestheater-tuebingen.de/}} führt in Tübingen regelmäßig größere und kleinere Inszenierungen von bekannten und unbekannten Stücken auf. Sehr zu empfehlen ist an dieser Stelle das Improtheater "Theatersport".	%TODO insert \link{}{}?

\subsubsection*{Kino}
Neben dem Unikino gibt es in Tübingen vier große Kinos: Das Kino Blaue Brücke, das Kino im Museum, Atelier und das Arsenal. In der blauen Brücke und dem Arsenal werden eher die großen Hollywoodfilme gespielt, während im Museum auch immer wieder kleinere und unbekanntere Filme gezeigt werden. Dort finden auch jährlich die Französischen Filmtage statt.

\subsubsection*{Clubhausfest}
Eine Institution des Tübinger Nachtlebens ist das jeden Donnerstag stattfindende Clubhausfest im Clubhaus (Haltestelle Uni/Neue Aula). Jede Woche richtet eine andere Fachschaft oder Gruppierung in diesem Gebäude eine große Party aus. Besonders lohnenswert: Das Clubhausfest der Informatik!

\subsubsection*{Boulderzentrum}
Ein Stück außerhalb von Tübingen Richtung B27 unterhält der Deutsche Alpenverein (DAV) ein Boulder- und Kletterzentrum. Wer sich mal ordentlich auspowern möchte, ist hier genau richtig. An zahlreichen Kletterwänden unterschiedlichen Schwierigkeitsgrad kann man seine Kletterfertigkeiten verbessern. Mitglieder des DAV zahlen zusätzlich nur einen verringerten Eintrittspreis. Weitere Infos unter \url{http://b12-tuebingen.de/}.	%TODO insert \link{}{}?
