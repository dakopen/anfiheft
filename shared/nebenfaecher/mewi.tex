In Medienwissenschaften werdet ihr in die Kerngebiete der Forschung und Analyse von Medien eingeführt. Es müssen Veranstaltungen aus vier Modulen (Grundlagen, Forschung \& Analyse, Lehrredaktion, Praxis und Technik) belegt werden. Dazu stehen Veranstaltungen wie z.B. "`Medien und Urheberrecht"', "`Medienforschung"', "`Medienanalyse"', "`Medienwissenschaftliche Theorien und Methoden"', "`Mediensysteme"' oder "`Text- und Mediendesign"' zur Verfügung.

Eine detaillierte Beschreibung der einzelnen Module sind im Modulhandbuch der Medienwissenschaft für Informatik (B.Sc.) der Neuphilologischen Fakultät auffindbar. Das Modulhandbuch für Informatik bietet einen Grundlegenden Überblick über zu belegende Veranstaltungen.

Infos zu den Vorlesungen, Seminaren und Terminen gibt es auf der Webseite\footnote{\url{http://www.medienwissenschaft.uni-tuebingen.de/}} der Medienwissenschaften unter "`Nebenfach Informatik"'.


%Natürlich könnt ihr euch auch an uns oder, wenn ihr ganz verbindliche Auskünfte wollt, an Dipl. Medienwiss. Felix  Reer Sprechstunden: Mi., 10:00-12:00 Uhr, Raum 211  \footnote{\email{felix.reer@uni-tuebingen.de }} wenden.
